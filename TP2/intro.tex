Le progrès établi dans les domaines scientifiques et technologiques a permis l’apparition et le développement de nouveaux objets, similaires à ceux que nous connaissons actuellement, mais avec la faculté supplémentaire d’être connectés à Internet. Ce déploiement d’Internet aux objets physiques est appelé Internet Of Things, ou IoT.

IoT représente donc une nouvelle manière de voir notre monde d’aujourd'hui, à savoir un monde où les objets communiquent entre eux, et sont interconnectés. Plus précisément, IoT est un « réseau de réseaux qui permet, via des systèmes d’identification électronique normalisés et sans fil, d’identifier et de communiquer numériquement avec des objets physiques afin de pouvoir mesurer et échanger des données entre les mondes physiques et virtuels » \cite{benghozi2009internet}
\\

Bien évidemment, IoT a vu le jour afin de répondre à des problématiques importantes, notamment avoir la possibilité de fournir une mesure en continue de certaines variables, tout en minimisant les coûts technologiques, de maintenance, et en ressources humaines.
\\

L’IoT peut ainsi être appliqué au Smart Health, à savoir être utilisé dans les soins cliniques où les patients, dont le statut physiologique nécessite une attention particulière, peuvent être surveillés en permanence en utilisant des outils non invasifs et dirigés par l’IoT. Cela nécessite par conséquent des capteurs sans fils pour recueillir de l'information physiologique précise et complète, et implique d’utiliser des routeurs hybrides, permettant d'identifier les capteurs ne pouvant pas accueillir d'adresse IP, des passerelles (gateways) et le Cloud pour analyser et stocker les informations, puis pour envoyer (de manière sans fil) les données analysées aux aides-soignants et aux médecins pour une analyse et des examens plus approfondis.

Ces techniques permettraient d’améliorer la qualité des soins grâce à une surveillance constante et de réduire le coût des soins en éliminant la nécessité d’avoir activement et systématiquement un aide-soignant pour collecter et analyser les données.

En outre, la technologie peut être utilisée pour la surveillance à distance à l'aide de petites solutions connectées sans fil à travers l'IoT. Ces solutions peuvent être utilisées pour capturer des données en toute sécurité sur la santé des patients à partir d'une variété de capteurs, pour appliquer des algorithmes complexes pour analyser les données, puis pour les partager grâce à la connectivité sans fil avec les professionnels médicaux qui peuvent faire des recommandations de soins appropriées.

Il faut donc permettre aux applications de surveillance dans le domaine de la santé de collecter des données provenant de capteurs, de fournir un support pour les interfaces et les affichages de l’utilisateur, d’avoir une connexion au réseau pour l’accès aux services de l’infrastructure, et bien évidemment d’être robuste, durable, précis, fiable, et faible en consommation d’énergie \cite{vermesan2014internet}.
\\

Pour en revenir aux objectifs principaux de tels hôpitaux « intelligents », ils se résument à améliorer la qualité de vie des personnes qui ont besoin de surveillance permanente, à diminuer les barrières concernant la surveillance d’importants paramètres de santé, à éviter les coûts et les efforts de santé inutiles, et à fournir un support médical juste et approprié au bon moment.
\\

Dans le cadre de nos laboratoires, nous nous intéresserons donc aux hôpitaux, et plus particulièrement aux services de soins intensifs qui nécessitent une surveillance permanente de patients (qu’ils soient animés ou inanimés), et l’utilisation de hautes technologies de mesure. Le but de nos laboratoires sera alors d’apporter une solution IoT face à la congestion des hôpitaux, et plus précisément dans les services de soins intensifs, en distinguant plusieurs cas d’utilisation (patients inanimés ou non). Notre premier laboratoire consistera à décrire les différents dispositifs qu’il est possible d’utiliser dans notre cas, ainsi que les protocoles de réseau et de communication qu’il faudra appliquer pour permettre à nos dispositifs de communiquer entre eux. Aussi, nous établirons en dernier lieu les problématiques liées à notre solution.
