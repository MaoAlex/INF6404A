Dans l'Internet des objets, tout comme dans tout système réparti, une communication entre les différentes composantes est
nécessaire. Or les données issues de divers senseurs seront probablement sous des formats variés. Il est donc obligatoire de
traduire ces données sous un autre format, qui serait le format de référence pour le reste de l'application. Cette traduction
serait à la charge du middleware afin de fournir un modèle uniforme aux futurs développeurs. Cette considération reste vraie
quel que soit l'endroit (moniteur, smartgateway) où le middleware sera déployé: la smartgateway devra récupérer des données sur le
patient depuis le cloud et les transmettre au médecin. Cette transmission nécessitera un format. De plus, il serait souhaitable
que des mécanismes soient déjà implémentés afin de générer des classes (ou tout autre abstraction utile pour le programmeur) afin
que le dit programmeur puisse directement manier ces classes sans avoir de la lecture et de l'écriture au sein de ce format.
\newline

De plus, ce format devrait être le plus léger possible. En effet, il conditionne l'apparence de ce qui circulera sur le
réseau. De ce fait, un format trop verbeux ralentirait la propagation des données et pourrait éventuellement être source de
congestion. Toutefois, ce format doit aussi être extensible. On veut pouvoir ajouter de nouvelles données si on introduit un
nouveau capteur par exemple. Enfin, ce format doit être indépendant de la plateforme ou du langage utilisé.  
\newline

Pour un tel format, notre choix s'est porté pour le \textit{protocol buffer}. En effet, il remplit tous les requis spécifiés
ci-dessus, mais a de plus des propriétés non négligeable. En particulier, ce format binaire est compatible vers l'avant et vers
l'arrière et peut traiter des messages même s'ils contiennent des champs inconnus. Cette particularité permet de faire cohabiter
différentes version d'un même logiciel. Or le problème de la mise à jour dans les grands est un problème compliqué. De plus, dans
le cadre d'un hôpital, il est impensable d'arrêter tous les systèmes pour procéder à une mise à jour globale. Cette mise à jour
sera donc incrémentielle. Or, il est tout aussi impensable que deux machines n'ayant pas la même version de logiciel ne puissent
pas échanger de message. 
\newline

Pour présenter de façon succincte le \textit{protocol buffer}, on pourra dire que c'est un format développé par \textit{Google}
avec lequel chaque message est composé de clés, types et valeurs.a Où la clé est le numéro du champ dans la spécification.

Notons que lorsqu'on parle de différentes versions, il faut, pour que la compatibilité arrière soit assurée, que les anciens
champs du message ne soit pas supprimés. En revanche, l'introduction de nouveaux champs ne posera pas de problème si les protocol
buffer sont utilisés. (Les champs inconnus seraient simplement ignorés).
\newline

Par ailleurs, la taille des messages envoyés est particulièrement optimisée. Nous avons déjà dit qu'il s'agissait d'un format
binaire, mais nous avions omis de préciser que les entiers (taille, poids, âge) sont transmis via des \textit{varint}.
C'est-à-dire des entiers de tailles variables: plus l'entier est petit moins il prend de place.
