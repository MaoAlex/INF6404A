Les exigences de Middleware abordées ici sont universelles, car applicables à n'importe quel Middleware, et concerne les exigences architecturales des Middlewares.
\\

L'architecture de notre Middleware doit tout d'abord présenter une abstraction de la programmation, c'est-à-dire fournir une API (Interface de Programmation Applicative) pour les développeurs d'application. C'est une exigence fonctionnelle importante pour tout middleware. Pour le développeur de l'application ou du service, des interfaces de programmation de haut niveau ont besoin d'isoler le développement des applications ou des services provenant des opérations prévues par les infrastructures IoT hétérogènes et sous-jacentes. Le niveau d'abstraction, le paradigme de programmation, et le type d'interface doivent tous être pris en considération lors de la définition d'une API.

L'architecture de notre Middleware doit aussi être interopérable. Un middleware devrait en effet fonctionner avec des appareils, des technologies, ou des applications hétérogènes sans effort supplémentaire de la part du développeur de l'application ou du service. Les composants hétérogènes doivent être en mesure d'échanger des données et des services. L'interopérabilité dans un middleware peut être observée à partir du réseau, des perspectives sémantiques et syntaxiques, et chacun doit être pris en charge dans un IoT.

Notre Middleware doit être basé sur les services. Une architecture de middleware devrait être basée sur les services afin d’offrir une grande flexibilité lorsque les fonctions nouvelles et avancées doivent être ajoutées au middleware IoT. Un middleware basé sur les services fournit des abstractions pour le matériel sous-jacent complexe à travers un ensemble de services (par exemple la gestion des données, la fiabilité, la sécurité) nécessaires pour les applications.

L'architecture de notre Middleware doit être adaptable. En effet, un middleware doit être adapté afin qu'il puisse évoluer pour s'accorder à des changements dans son environnement. Dans l'IoT, le réseau et son environnement sont susceptibles de changer fréquemment. En outre, les demandes ou le contexte au niveau de l'application sont également susceptibles de changer fréquemment. Afin d'assurer la satisfaction des utilisateurs et l'efficacité de l'IoT, un middleware doit s’adapter dynamiquement ou s'ajuster à toutes ces variations.

Un middleware doit être context-aware (attentif au contexte), c'est-à-dire conscient du contexte des utilisateurs, des dispositifs, et de l’environnement et utiliser ce contexte pour des offres de services efficaces et essentiels pour les utilisateurs.

Notre middleware doit bien évidemment présenter une certaine autonomie. Les appareils, les technologies, ou les applications sont des participants actifs dans les processus de l'IoT et ils devraient pouvoir interagir et communiquer entre eux sans intervention humaine directe. L'utilisation de l'intelligence, y compris des agents autonomes, de l’intelligence embarquée, et des approches prédictives et proactives dans le middleware peut satisfaire cette exigence.

Enfin, comme nous le montre en partie la Figue 2, notre Middleware doit être distribué. En effet, les applications, les dispositifs, et les utilisateurs sont susceptibles d'être distribués géographiquement, et donc une mise en oeuvre d’une vue ou d’un middleware centralisé ne sera pas suffisant pour supporter de nombreux services ou applications distribuées. Une implémentation d'un middleware doit prendre en charge les fonctions qui sont distribuées à travers l'infrastructure physique de l'IoT.