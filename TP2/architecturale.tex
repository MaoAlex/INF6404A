\paragraph{}
Le middleware requiert la prise en compte d'un certain nombre de problématiques architecturales qui ont influé sur nos choix. Nous en donnons la description ci-dessous :\\
\begin{itemize}
\item \textbf{\textit{Abstraction de programmation :}} Il est important d'offrir, dans n'importe quel middleware (pas uniquement ceux concernant IoT), des interfaces de programmation (API) afin de proposer aux développeurs de services ou d'applications des interfaces de programmation haut niveau qui permettent d'abstraire l'hétérogénéité des systèmes sous-jacents et ainsi d'isoler le développement des applications et des services de ces considérations bas-niveau.\\
\item \textbf{\textit{Interopérabilité :}} En raison de la grande diversité des protocoles, objets, applications qui sont très hétérogènes, particulièrement dans le domaine de l'IoT, un middleware se doit de permettre aux développeurs d'application ou de services de ne pas se préoccuper de ses problématiques et de se concentrer sur les données et services qu'il est possible d'obtenir et d'échanger avec les composantes du système. Cette interopérabilité doit être prise en compte au niveau des réseaux, de la sémantique et de la syntaxe des requêtes/ordres.\\
\item \textbf{\textit{Basé sur les services :}} Une architecture de middleware devrait être découpée en modules s'appuyant sur des services qui permettent une grande flexibilité et un ajout facilité de nouvelles fonctionnalités (soit plus avancées, soit récentes). Ce requis intervient également dans la prise en compte des deux requis précédents, car les services favorisent l'abstraction de programmation et l'interopérabilité en substituant le point d'accès qu'est le service aux composantes du système.\\
\item \textbf{\textit{Adaptabilité :}} Un middleware doit être évolutif et permettre de s'adapter à des nouvelles conditions d'environnement technique, technologique, ... Dans l'IoT spécifiquement, le réseau et l'environnement du middleware sont susceptibles de changer rapidement et fréquemment. Le contexte au niveau de l'application ou les requêtes possibles sont également susceptibles d'évoluer selon les changements apportés. Il est alors important que le middleware puisse s'adapter dynamiquement, ou au moins s'ajuster à ces variations.\\
\item \textbf{\textit{Connaissance du contexte (Context-awareness) :}} Le middleware IoT doit être conscient du contexte dans lequel les données, les événements, les ordres, ... ont été émis. Ceci comprend une conscience des dispositifs, utilisateurs et environnement du middleware ainsi que l'utilisation adéquate de ces informations afin d'offrir des services précis, efficaces et utiles pour l'utilisateur.\\
\item \textbf{\textit{Autonomie :}} Les appareils, technologies et applications intervenant dans l'IoT sont des agents actifs et intelligents, qui doivent pouvoir communiquer, échanger des informations  et interagir sans nécessité d'intervention humaine directe.\\
\item \textbf{\textit{Distribuée :}} Les applications, dispositifs et utilisateurs sont susceptibles d'être distribués géographiquement, et donc un middleware centralisé peut ne pas être suffisant pour supporter les services ou applications distribuées qui s'appuient sur lui. Une architecture de middleware pour IoT doit donc prendre en charge des fonctionnalités distribuées à travers l'infrastructure physique IoT.\\
\end{itemize}