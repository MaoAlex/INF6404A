Dans le premier rapport nous avions présenté notre solution IoT orientée Smart Health, afin de résoudre le problème de congestion des hôpitaux dans les services de soins intensifs. Plus précisément, nous avions proposé les dispositifs et les protocoles de communication et de réseau de notre solution. Enfin, nous avions abordé les enjeux liès à la QoS, et à la sécurité et la privacité de notre solution.
\\

Le présent rapport a permis quant à lui de présenter le rôle du middleware dans la réponse aux problématiques de congestion de l'hôpital dans les services d'urgences (donc dans le cadre de notre solution). Les exigences de ces dernières ont été séparées en trois catégories, à savoir les exigences fonctionnelles, non-fonctionnelles et architecturales. Afin de remplir les différentes fonctionnalités attendues du système, différents modules au sein de notre Middleware ont ainsi été identifiés. Chaque module serait destiné à remplir une fonctionnalité majeure, et à respecter une ou possiblement plusieurs exigences du middleware. Par ailleurs, nous avons présenté une conception répartie pour le middleware, conception qui peut se résumer par la présence de middlewares distincts au niveau du moniteur et de la passerelle (gateway), nous proposons des middlewares qui pourraient opérer de manière localiser comme dans un hôpital mais qui pourraient aussi être intégrer à une plateforme IoT{IBM}.

Dans ce rapport, nous avons également mis l'accent sur les considérations sécuritaires et de confidentialité à prendre en compte lors de la conception d'un tel système, ou plus précisément d'un tel middleware.
\\

L'objet du prochain rapport sera d'établir et de conceptualiser l'architecture de notre solution IoT orientée Smart Health, incluant la manière de stocker les données dans le Cloud, et les différentes façon dont notre solution sera exploitable à travers une application. Ce dernier rapport résumera la conception générale de notre solution.
