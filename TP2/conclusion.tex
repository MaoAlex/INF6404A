Dans le premier rapport nous avions présenté notre solution IoT orientée Smart Health, afin de résoudre le problème de congestion des hôpitaux dans les
services de soins intensifs. Plus précisément, nous avions proposé les dispositifs et les protocoles de communication et de réseau
de notre solution. Enfin, nous avions abordé les enjeux liès à la QoS de notre solution.
\newline

Le présent rapport a permis de présenter le rôle du middleware dans la réponse aux problèmatiques mentionnées ci-dessus. Les
requis des ces dernières ont été séparés en trois catégories: fonctionnel, non-fonctionnel et architectural. Afin de remplir les
différentes fonctionnalités attendues du système, différents modules ont identifiés. Chaque module serait destiné à remplir une
fonctionnalité majeure. De plus, au sein de ces modules, nous avons les requis du middleware auxquels ils seraient liés.
Par ailleurs, nous avons présenté une conception répartie pour le middleware. Conception qui peut se résumer par la preèsence de
middleware distincts au niveau du moniteur et de la passerelle.
\newline

De plus, l'accent à été mis sur les considérations sécuritaires à prendre en compte lors de la conception d'un tel système.
