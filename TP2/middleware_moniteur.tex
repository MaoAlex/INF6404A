Dans cette partie, nous allons nous intéresser au middleware présent au niveau du moniteur. Afin de contextualiser, nous allons
rappeler les fonctionnalités principales de ce dernier. Premièrement, il a un rôle d'agrégateur. Il doit donc rassembler les
données issues de différents capteurs et les envoyer vers la passerelle intelligente via un unique message. 
\newline

Ce requis impose de pouvoir comprendre différents protocoles, tant au niveau physique que applicatif, ainsi que différents
formats. Il est en effet peu probable que tous les capteurs utilisent le même protocole pour leurs données sous un même format. Il
est également nécessaire de permettre l'évolutivité au sein du système afin que de futurs capteurs plus performants, mais
utilisant une nouvelle façon de communiquer, puissent être intégrés au système.
\newline

De plus, afin d'optimiser le temps du personnel médical, il est souhaitable que le moniteur et donc le middleware facilite la
découverte de capteurs ainsi que la récupération de leur contexte. Une fois les capteurs installés, le lien entre les données
reçues et l'identité du patient devrait être fait automatiquement (ou presque). Ces mêmes capteurs devraient être accessibles
depuis l'extérieur. En effet, pour certaines données relevées rarement, il est possible qu'un médecin souhaite les rafraîchir
plutôt que d'attendre la prochaine mise à jour. Il se peut également que le médecin désir donner un ordre à un des objets
intelligents. Pour cela, il faut que ces objets aient un nom, ou du moins soient adressables. En particulier, un nom du type
\textit{idPatient@typeDeCapteur} serait particulièrement pratique à utiliser.
\newline

Enfin, il est impératif, dans la mesure du possible, de sécuriser les communications au sein du système. Cela passe par un
contrôle d'accès au niveau du moniteur avec un mécanisme d'authentification. Mais aussi par une détection des pannes. En effet, il
est souhaitable que, de part l'importance des données fournies par ces capteurs, ils doivent être remplacés dès le premier indice
de dysfonctionnement. De manière générale, il est important de garantir une bonne disponibilité ainsi qu'une certaine fiabilité.

\begin{table}[h!]
    \centering
    \resizebox{\textwidth}{!}{  
    \begin{tabular}{|c|c|c|c|c|c|c|c|c|c|c|}
        \hline
        Modules & découverte ressources & scalabilité & interopérabilité & adaptabilité & fiabilité & disponibilité &  temps
        réel & basé sur les services &  context aware & autonome\\
        \hline
        \hline
        découverte des ressources & \checkmark & \checkmark & & \checkmark & & & & &&\\
        \hline
        détection panne & & & & & \checkmark & \checkmark & \checkmark &  \checkmark && \\
        \hline
        décision & & \checkmark & & \checkmark & \checkmark & \checkmark & & & & \checkmark \\
        \hline
        Gestion du contexte & & & && & &&& & \checkmark \\
        \hline
        émission d'événements & & & \checkmark & \checkmark & & \checkmark & & & \checkmark & \\
        \hline
        gestion des différents protocoles physiques & & \checkmark & && & & & & & \\
        \hline
        nommage des capteurs & & \checkmark & \checkmark & \checkmark && & & & & \\
        \hline
        agrégation de données et centralisation & & \checkmark & \checkmark & \checkmark & & & & & & \\
        \hline
        gestion des différents protocoles applicatifs & & \checkmark & \checkmark & \checkmark & & & & & &\\
        \hline
    \end{tabular}
    }
    \caption{Utilisation des services du middleware par les différents modules}
    \label{tab:monitor_middleware}
\end{table}

\begin{table}[h!]
    \centering
    \resizebox{\textwidth}{!}{  
    \begin{tabular}{|c|c|c|c|c|}
        \hline
        Modules & gestion ressources & gestion données & gestion événements & gestion du code \\
        \hline
        \hline
        découverte des ressources & \checkmark & & & \\
        \hline
        détection panne & \checkmark & & & \\
        \hline
        décision & & & &\\
        \hline
        Gestion du contexte & & \checkmark & \checkmark & \\
        \hline
        émission d'événements & & \checkmark & \checkmark & \\
        \hline
        gestion des différents protocoles physiques & & & & \checkmark \\
        \hline
        nommage des capteurs & \checkmark & & & \\
        \hline
        agrégation de données et centralisation & & \checkmark & & \\
        \hline
        gestion des différents protocoles applicatifs & & & & \checkmark \\
        \hline
    \end{tabular}
    }
    \caption{Utilisation des services de gestion du middleware par les différents modules}
    \label{tab:monitor_middleware_gestion}
\end{table}
Les tableaux \ref{tab:monitor_middleware} et \ref{tab:monitor_middleware_gestion} présentent les modules sur la première colonne avec les exigences du middleware sur la
première ligne.
\newline

Par ailleurs, il serait aussi nécessaire de présenter une API simple d'utilisation au niveau du middleware afin de faciliter la
programmation de nouvelles fonctionalité.
