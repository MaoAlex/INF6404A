Les exigences fonctionnelles du Middleware correspondent aux services et aux fonctions qu’un middleware doit fournir, et sont donc représentées par la découverte des ressources, et toute forme de gestion (gestion des données, des ressources, des événements et du code).
\\

Tout d'abord, une des exigences les plus significatives du Middleware est la \textbf{\textit{découverte des ressources}}. Les ressources IoT comprennent les dispositifs hétérogènes de matériel (par exemple les étiquettes RFID, les capteurs, et les smartphones), la puissance et la mémoire des dispositifs, analogues à des dispositifs de conversion numérique (A/D), le module de communication disponible sur ces appareils, les informations au niveau de l'infrastructure ou du réseau (par exemple la topologie du réseau et des protocoles), et les services fournis par ces dispositifs. Les hypothèses relatives aux connaissances globales et déterministes de la disponibilité de ces ressources ne sont pas valides, car l'infrastructure et l'environnement de l'IoT est dynamique (les dispositifs, et leur nombre seront différents en fonction du patient qui est reçu). Ainsi, l'intervention humaine pour la découverte de ressources est infaisable et, par conséquent, une condition importante pour la découverte de ressources est qu'elle doit être automatisée. Par ailleurs, lorsqu'il n'y a pas de réseau d'infrastructure, chaque appareil doit annoncer sa présence et les ressources qu'il offre.

\textbf{\textit{La gestion des ressources}} est une exigence importante qui entre en jeu naturellement après la découverte des ressources du Middleware, car elle se réfère à la qualité de service (QoS ou Quality of Service). En effet, une QoS acceptable est attendue pour toutes les applications, et dans un environnement où les ressources ayant un impact sur la qualité de service sont limitées, comme l'IoT, il est important que les applications soient fournies avec un service qui gère ces ressources. Cela signifie que l'utilisation des ressources doit être surveillée, que les ressources allouées ou provisionnées le soient de manière équitable, et que les conflits de ressources soient résolus.

\textbf{\textit{La gestion des données}} est également une caractéristique primordiale au sein du Middleware. Dans l'IoT, les données se réfèrent principalement aux données détectées ou toute autre information d'infrastructure de réseau d'intérêt pour les applications. Ainsi, un middleware IoT doit fournir des services de gestion des données pour les applications, y compris l'acquisition de données, le traitement des données (incluant le prétraitement des données), et le stockage de données. C'est une exigence importante à respecter dans notre cas car il faut récupérer les données des différents dispositifs, puis les traiter, et les interpréter, ce que nous décrirons par la suite à travers certains modules de notre Middleware.

Comme nous avons un système qui dépend des événements, la gestion de ceux-ci est aussi une exigence à considérer dans notre solution IoT. Il y a en effet potentiellement un grand nombre d'événements générés dans les applications IoT, qui devraient être gérés comme une partie intégrante d'un middleware IoT.  \textbf{\textit{La gestion des événements}} transforme les événements observés simples en des événements significatifs. C'est donc une exigence importante dans notre cas pour gérer les alertes, que ce soit au niveau des pannes des dispositifs, mais aussi au niveau des urgences (qui nécessitent l’intervention de personnel soignant).

Enfin, en temps qu'exigence fonctionnelle, il est nécessaire d'avoir une certaine \textbf{\textit{gestion du code}}, car le déploiement du code dans un environnement IoT est difficile, et doit être directement pris en charge par le Middleware. En particulier, les services d'allocation de code et de migration de code sont nécessaires, et il est possible que les technologies en terme de dispositifs et de protocoles de communication et de réseau évoluent.