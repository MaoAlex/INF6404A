Les middlewares ont un rôle important dans tout système d’IoT, et ce sont eux qui vont faire la transition entre les objets connectés et les couches supérieures, notamment celles de stockage. C’est pourquoi nous avons mis en évidence les différentes exigences non fonctionnelles du middleware suivantes par rapport à notre système.
\\

\textbf{\textit{L'évolutivité}} correspond au fait qu'un middleware d’un système de l’IoT doit pouvoir évoluer pour pouvoir répondre à l’expansion du réseau et à l’ajout d’applications et de services. Dans notre cas, l’évolutivité du middleware doit pouvoir prendre en compte l’ajout d’éventuels capteurs, et différents objets connectés. Il doit par ailleurs laisser la possibilité de fournir des services supplé\-mentaires ainsi que des applications diverses en fonction des besoins.

Il faut permettre au middleware de respecter l'\textbf{\textit{exigence du temps réel}}. En effet, certains middlewares doivent pouvoir fournir des informations ou des services en temps réel lorsque la correction d'une opération qu’il prend en charge dépend non seulement de la correction logique, mais aussi du temps pendant lequel cette opération est réalisée. De nombreuses applications en temps réel vont utiliser les données fournies. L’envoi et la réception à temps des informations ou des services dans ces applications sont alors essentiels. Certains capteurs que nous avons proposés ont pour but de surveiller des données vitales sur le patient, par conséquent ces informations doivent pouvoir être consultées en temps réel car des délais de transmission peuvent avoir des conséquences graves voire mortelles sur un patient.

\textbf{\textit{La fiabilité}} d'un middleware est une exigence essentielle. Un middleware doit rester opérationnel pendant toute la durée d'une mission, même en présence de pannes. La fiabilité du middleware aide en dernier recours la réalisation de la fiabilité au niveau du système. Chaque composant ou service dans un middleware doit être fiable pour atteindre la fiabilité globale, ce qui inclut celles de la communication, des données, des technologies et des dispositifs de toutes les couches. Les signes vitaux qui sont surveillés imposent au middleware d’être capable de transmettre les données de façon fiable sans corruption de celles-ci, et cela toujours à cause des données qui sont surveillées.

Il y a la nécessité également d'introduire \textbf{\textit{la disponibilité}} au sein de notre middleware. En effet, même s'il y a une défaillance quelque part dans le système, le temps de récupération et la fréquence de défaillance de celui-ci doivent être suffisamment petits pour obtenir la disponibilité souhaitée. Les exigences en matière de fiabilité et de disponibilité doivent travailler ensemble pour assurer la plus haute tolérance aux pannes nécessaire depuis une application. Le middleware doit pouvoir être capable de détecter si l’un de nos capteurs est en panne, et pour cela, il va faire des requêtes à intervalles réguliers des signes vitaux et en cas d’absence de réponse pendant un temps prédéfini, une alerte sera envoyée au personnel médical pour vérifier l’origine de l’erreur. Dans le domaine de la santé, où la vie des patients est en jeu, nous ne pouvons pas nous permettre de laisser le patient sans surveillance médicale.

\textbf{\textit{La sécurité}} et \textbf{\textit{la privacité}} sont des exigences à ne pas oublier dans un tel système qu'est le nôtre. La sécurité doit être prise en compte dans tous les blocs fonctionnels et non fonctionnels. Nous avons différentes couches de sécurité dans les parties de notre architecture. Tout d’abord au niveau de l’identification des appareils connectés, nous proposons un système d’identification des appareils auprès du moniteur intelligent, ce qui évitera la connexion de capteurs non souhaités, et nous pourrons aussi dans l’autre sens connecter un capteur qu’à un seul moniteur. Au niveau de nos moniteurs intelligents, il y aura la mise en place d’un système de chiffrement des données pour l’envoi à la couche supérieure pour préserver la confidentialité des données envoyées du patient.

\textbf{\textit{La facilité de déploiement}} est également une exigence que nous visons au sein de notre middleware, car le déploiement ne devrait pas exiger des connaissances spécialisées, et les procédures d'installation et de configuration compliquées devraient être évitées. Dans notre cas l’ajout de capteur se fera à travers une interface simple sur le moniteur intelligent qui identifiera cet appareil au réseau mère. Et ainsi, l’ajout de nouveaux dispositifs se fera de manière aisée.