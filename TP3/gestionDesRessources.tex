La gestion des ressources est une discipline présente dans tous les domaines. Le domaine médical ne fait pas exception à cette
règle. En effet, un certain équipement est nécessaire pour traiter les malades. Ce matériel peut se dégrader avec le temps ou
suite à un accident. A terme il faudra donc le remplacer. Afin de procéder aux achats, il est préférable de faire un inventaire,
cela permet de faire des commandes globales et donc d'économiser de l'argent. De nombreux commerces ferment durant un inventaire
afin de simplifier l'opération, toutefois cette solution n'est pas envisageable dans le cas des hôpitaux en général, en
particulier cela serait inacceptable dans le service des urgences.
\newline

Le système doit donc permettre de faire un inventaire. Il serait préférable qu'il permette non seulement de connaître le nombre de
dispositifs selon le type de ces derniers, mais aussi le nombre de dispositifs d'un certain modèle. En effet, un modèle pourrait
posséder plus de fonctionnalités, ce qui le rendrait indispensable pour certaines situations, ou du moins plus pratique. Intégrer
cette fonctionnalité à notre système plutôt que de se basé sur un logiciel/mécanisme déjà existant a quelques avantages
non négligeables. En particulier, sous l'hypothèse que chaque capteur possède un trait unique, il est possible d'automatiser une
partie du processus. Comme exemple de trait distinctif, on peut penser aux adresses mac que des constructeurs donnent à leurs
produits. Le principe de fonctionnement serait le suivant: lorsque le capteur se déclare à son moniteur, ce dernier informerait le
système (via un message à la passerelle intelligente) de la connexion qui vient de s'établir. La passerelle pourrait alors
informer un serveur interne qu'un capteur ayant un tel id vient d'initier une connexion. Le système vérifie s'il connaît cet id,
si ce n'est pas le cas il envoie un message à la passerelle demandant plus d'information. Il sera alors de la responsabilité de la
passerelle de le renseigner.
\newline

Par ailleurs,en plus de connaître le nombre d'appareils, il peut être utle de pouvoir tracer des courbes récapitulant
l'utilisation par type d'appareil. En effet, la possibilité de calculer diverses statistiques permettrait de mieux anticiper les
besoins. En outre, cela est indispensable pour la prise de décision. En l'absence de telles statistiques, il est difficile de
décider le nombre d'appareils à acheter.
