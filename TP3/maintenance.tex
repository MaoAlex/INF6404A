Un hôpital doit avoir un équipement le plus fiable possible. En effet, une simple défaillance peut causer la mort du patient, ou
à défaut de graves dommages. Il est donc important de choisir prudemment les équipements à utiliser. Malheureusement l'équipement
sans faille n'existe pas et lorsqu'une panne se produit, il est capital de pouvoir déterminer l'origine de la défaillance. Pour
cela la présence de logs est indispensable à tous les niveaux. 
\newline

Au niveau du moniteur, il faut pouvoir savoir quel type de capteur s'est connecté (marque, modèle) afin d'identifier d'éventuelles
incompatibilités. Il faut également enregistrer les moments où le moniteur a donné un ordre, ainsi que le ce qu'était cet ordre.
Les réponses reçues sont aussi à conserver.
\newline

La passerelle intelligente (parfois appelée smart gateway) doit également avoir un historique des messages/demandes qu'elle a
reçus. Allié avec les logs issus d'un moniteur cela peut permettre d'identifier des problèmes réseaux, des interblocages et bien
d'autres dysfonctionnements.
\newline 

Pour des raisons similaires, des logs doivent être aussi présents au niveau du stockage interne et du cloud. En plus de permettre
une meilleure compréhension des pannes, ils pourraient avoir des applications au niveau de la sécurité. Entre autre, ils seraient
un moyen simple pour s'assurer que personne n'a essayé d'accéder à données qui ne leur sont pas destinées.
\newline

La présence de logs, impose de pouvoir les gérer. En effet, ces logs risquent d'être volumineux, il faut donc avoir la
possibilité de trier ces derniers afin de ne récupérer que les données pertinentes. Mettons que l'on est l'intégralité des logs
d'une passerelle intelligente, il serait souhaitable de pouvoir extraire toutes les données relatives à un moniteur.
