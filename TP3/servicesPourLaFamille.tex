Le système que nous proposons dans le présent document possédera des services accessibles depuis l'extérieur. Parmi ces
fonctionnalités, certains seront dédiées à la famille des patients. Ces services auraient, bien entendu un contrôle d'accès très
strict. Une possibilité que nous avons envisagée est l'envoie d'un sms aux numéros à prévenir en cas d'urgence d'un patient. Ce
sms contiendrait un mot de passe et un identifiant valides que durant la période d'hospitalisation. Le destinataire pourrait alors
se connecter à un site web (via une connexion sécurisée) où il pourrait trouver différentes informations et différents conseils.
\newline

Les informations visibles seraient la localisation de l'hôpital, comment s'y rendre ainsi que la chambre du patient. On pourrait
ajouter à cela les horaires des visites. Si l'on désire pousser cette idée plus loin, on peut imaginer un système où la famille
pourrait prendre rendez-vous sur un calendrier. S'il y a trop de visiteurs sur la période choisie, la demande de visite sera
refusée et l'utilisateur sera invité à choisir un autre moment. Ce système permettrait de mieux étaler les visites, évitant ainsi que le
patient ait trop de visiteurs à un certain moment, et aucune à d'autres moments. Cela permettrait aussi d'éviter que des visiteurs
viennent à une période où un opération est prévue. Ce qui les obligeraient à rebrousser chemin. En revanche les informations
médicales du patient ne seraient pas affichée, d'une part pour éviter une certaine atteinte à la vie privée, et d'autre part pour
éviter des mouvements de panique du côté des utilisateurs. De manière générale, les consignes à respectées lors d'une visite
seraient visibles sur ce site. Et certaines informations sans lien direct avec l'hôpital mais utiles pour le confort des patients
pourraient être ajoutées. Connaître l'adresse de l'hôtel (ou équivalent) le plus proche serait, par exemple, un plus pour les
visiteurs.
\newline

En plus des informations citées plus haut, nous aurions également des conseils à l'intention des personnes qui rendent visite aux
patients. Ces conseils pourraient être personnalisés selon le patient. A titre d'exemple, si leur traitement leur interdit un certain type d'aliments
cela serait précisé. 
\newline

Pour conclure cette partie, on notera qu'un tel système permettrait de réduire la charge du service public de l'hôpital, car de
nombreux services qui étaient auparavant à leur charge seraient maintenant remplis par le site internet. De plus, il permettrait
d'améliorer l'expérience du visiteur en lui fournissant des informations et services utiles et disponibles à toutes heures.
